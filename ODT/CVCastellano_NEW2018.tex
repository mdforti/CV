Resume

Mariano Daniel Forti

\section{Introduction}\label{introduction}

Actualmente me desempeño como investigador en la Comisión Nacional de
Energía Atómica Argentina. Llevo a cabo mis actividades en la División
Aleaciones Especiales (DAE), donde empecé a trabajar como especialista
en cálculos DFT en 2017. Mis investigaciones se centran en la
estabilidad mecánica de sistemas interfaciales. Mayormente he trabajado
sobre el sistema Hierro / Magnetita y más recientemente en el sistema
Circonio /Circonia. También participo en otras actividades de la DAE,
por lo que participo en estudios sobre defectos puntuales en ZrO2.

Desde mi graduación como Ingeniero, también trabajo como ayudante de
primera en una materia avanzada en el Instituto Sabato, donde el
objetivo principal es la implementación computacional del Método de
Elementos Finitos.

En Septiembre de 2017 completé los requerimientos para obtener el
Doctorado en Cienca y Tecnología de la Universidad de General San
Martín. Mi título de grado es Ingeniero en Materiales, y desde el
Trabajo Final de Ingeniería he trabajado en cálculos DFT basados en VASP
aplicados a sistemas superficiales e interfaciales.

Luego de obtener mi doctorado en 2017, he estado colaborando en la
Fundición de la DAE con los preparativos para la calificación de la
fabricación de aleaciones utilizadas en los componentes de seguridad en
las plantas de energía nuclear. Al mismo tiempo, asisto técnicamente al
personal de la fundición para la optimización de los procesos y gestión
de la calidad. Otras responsabilidades en la DAE también incluyen el
mantenimiento y administración de dos clústers de computadoras de altas
prestaciones (pequeños, HPC) propios de la DAE. Por lo tanto, durante mi
carrera he adquirido una amplia experiencia en la administración de
sistemas Linux para escritorio y para HPC. Poseo conocimientos de
programación en general, haciendo uso diario de los lenguajes Bash,
Fortran, Python, php y javascript. Al mismo tiempo utilizo herramientas
como Matlab, Octave, vim, OriginLab y gnuplot. En particular, soy
entusiasta y usuario de herramientas libres y de código abierto como el
entorno de escritorio KDE y el paquete de oficina LibreOffice.

\section{Experiencia Profesional}\label{experiencia-profesional}

Febrero 2017 -- Actual. Comisión Nacional de Energía Atómica.
Departamento de Materiales. Grupo de Aleaciones Especiales. Supervisor:
Dra. Paula Alonso pralonso@cnea.gov.ar). Investigador. Cálculos DFT en
sistemas interfaciales Fe/Fe3O4, Zr/ZrO2. Defectos Puntuales en ZrO2.
Administración de sistemas Linux en computadoras de escritorio y
clústers para Computación de Altas Prestaciones. Uso diario de Bash,
Fortran y Python. Soporte científico a la Fundición de Aleaciones
Especiales,incluyendo gestión de la calidad.

Agosto 2010 -- Actual. Instituto Sabato (UNSaM-CNEA), Ayudante de 1ra.
``Modelización de Materiales y Procesos''. Profesor Titular: Ruben Weht
(ruweht@cnea.gov.ar). En este curso, los estudiantes hacen su propia
implementación de varios métodos numéricos para resolver ecuaciones
diferenciales, incluyendo el método de Elementos Finitos y Diferencias
Finitas.

Agosto 2010 -- Enero 2017. Comisión Nacional de Energía Atómica.
Departamento de Materiales. Grupo de Aleaciones Especiales. Beca
Doctoral, ``Estudios Ab Initio sobre adherencia en interfases Hierro /
Oxidos de Hierro''. Directora: Dra. Paula Alonso (pralonso@cnea.gov.ar).
Desempeño tareas de Investigación y Desarrollo, Administración de
Sistemas Linux, tanto computadoras de escritorio como clusters para
computación de alta performance (HPC). Uso cotidiano de herramientas
programadas en Bash, Fortran, Python.
agusto 
agusto 

Agosto 2006-Agosto 2010. Instituto de Tecnología Jorge Sabato (ITJS,
UNSAM-CNEA). Ingeniería en Materiales. Beca de dedicación exclusiva.

Enero 2010 -- Julio 2010. Trabajo de Seminario de Ingeniería en
Materiales. ``Department of Chemical Engineering'', Texas A\&M
University, USA. ``Estudios Ab-Initio sobre Carburación en Aleaciones de
Base Fe3Al''. Directora: Dra. Perla Balbuena (balbuena@tamu.edu).

Febrero 2008. Pasantía en el Departamento de Física de CNEA. ``Síntesis
y Propiedades Mecánicas de compuestos Manganita/Polímero''. Referencia:
Griselda Polla (grispoll@cnea.gov.ar).

Febrero 2007. Pasantía en el Grupo de Materia Condensada, CNEA.
``Propiedades eléctricas y magnéticas de compuestos
Manganita/Polímero''. Referencia: Joaquín Sacanell
(sacanell@cnea.gov.ar).

\section{Participación en Proyectos de
Investigación}\label{participaciuxf3n-en-proyectos-de-investigaciuxf3n}

Febrero 2017 -- Febrero 2020. PICT-2015-2267, ``Buscando una nueva
aleación para el elemento combustible CAREM-25'', Comisión Nacional de
Energía Atómica, Agencia Nacional de Promoción Científica y Tecnológica,
Ministerio de Ciencia, Técnica e innovación Productiva.

Enero 2017 -- Diciembre 2019. Proyecto 80020160500046SM, ``Buscando una
nueva aleación para el elemento combustible CAREM-25''. Proyecto sin
financiamiento, Instituto Sabato, Universidad de San Martin, Comisión
Nacional de Energía Atómica, Ministerio de Ciencia, Técnica e Innovación
Productiva.

Agosto de 2011 -- Agosto de 2015. PICT-2011-1861, ``Integridad de
materiales en reactores nucleares: modelos atomístico/continuo aplicados
a interdifusión en combustibles dispersos y a fractura de la capa de
óxido en tuberías''. Ministerio de Ciencia, Técnica e Innovación
Productiva, ARS 262900.

Enero de 2011 -- Diciembre de 2012. Proyecto UNSAM C063, ``Pelicula de
óxido pasivante sobre hierro''. Proyecto Universidad Nacional de San
Martin , sin financiamiento.

Enero de 2011 -- Diciembre de 2011. ``Defectos constitucionales y
energía de migración de aluminio en UAl4''. Financiamiento Fundación
Balseiro y Comisión Nacional de Energía Atómica, ARS 10800.

Agosto 2010 -- Diciembre 2015. ``Estudios prospectivos e investigación y
desarrollo de tecnologías para nucleo-electricidad de cuarta
generación''. Financiamiento: Comision Nacinoal de Energía Atómica, ARS
400000.

Enero 2010 -- Diciembre 2012. ``Métodos computacionales aplicados al
estudio de propiedades físico-químicas de combustibles para reactores de
ivestigación y modelización de la migración de clusters de defectos en
materiaeles con interés tecnológico''. Financiamiento: CONICET, ARS
33700.

\section{Publicaciones}\label{publicaciones}

``Shear Behavior of Fe/Fe3O 4 interfaces''. Revista Materia V23-N2
(2018). Mariano Forti, Paula Alonso, Pablo Gargano, Gerardo Rubiolo.

``Properties of hexagonal Zr and tetragonal ZrO2 low index surfaces from
DFT calculations''. Revista Materia V23-N2 (2018). Paula Alonso, Pablo
Gargano, Laura Kniznik, Gerardo Rubiolo.

``Concentration of constitutional and thermal defects in UAl4 '' Journal
of Nuclear Materials 478 (2016) 74-82. Pablo Gargano, Laura Kniznik,
Paula Alonso, Mariano Forti, Gerardo Rubiolo.

``A DFT study of atomic structure and adhesion at the Fe(BCC)/Fe3 O4
interfaces''. Surface Science 647 (2016) 55--65. Mariano Forti, Paula
Alonso, Pablo Gargano,Perla Balbuena, Gerardo Rubiolo.

``Charge difference calculation in Fe/Fe3O4 interfaces from DFT
results''. Procedia Materials Science 8 (2015) pp 1066 -- 1072. Diego
Tozini, Mariano Forti, Paula Alonso, Pablo Gargano, Gerardo Rubiolo.

``Adhesion Energy of the Fe(BCC)/Magnetite Interface within the DFT
approach''. Procedia Materials Science 9 (2015) pp 612 -- 618. Diego
Tozini, Mariano Forti, Pablo Gargano, Paula Alonso, Gerardo Rubiolo.

``Ab-initio studies on carburization of Fe 3 Al based alloys''. Procedia
Materials Science 1 ( 2012 ) 191 -- 198. Mariano Forti, Perla Balbuena,
Paula Alonso.

``First principles study of U-Al system ground state''. Procedia
Materials Science 1 ( 2012 ) 514 --519. Laura Kniznik, Paula R. Alonso,
Pablo H. Gargano, Mariano D. Forti, Gerardo H. Rubiolo.

``Transition metals monoxides. An LDA+U study''. Procedia Materials
Science 1 ( 2012 ) 230 -- 234. Mariano Forti, Paula R. Alonso, Pablo H.
Gargano, Gerardo H. Rubiolo.

``Electric and magnetic properties of PMMA/manganite composites''.
Physica B 404 (2009) 2760-- 2762. Artale, C., Fermepin, S., Latino, M.,
Quintero, M., Granja, L., Sacanell, J., Polla G., Levy P.

\section{Participaciones en
Congresos}\label{participaciones-en-congresos}

Octubre 2018. 5th Nuclear Materials Conference, Elsevier -- IAEA, 14-18
October 2018, Seattle, USA. Presentación de Murales: 1)
``First-principles thermodynamic study of point-defect structure and
electrical conductivity in tetragonal non-stoichiometric zirconia
including lattice vibrations'', Gargano, Kniznik, Alonso, Forti,
Rubuiolo. 2) ``DFT Study of the Early Stages of Oxidation of the
Zr(1010) Surface'', F. Soto, M. Forti, P. Alonso, P. Gargano, L.
Kniznik, G. Rubiolo, P. Balbuena. 3) ``A DFT study of the resistance to
traction and shear loads of the Fe(BCC) / Fe3O4 interface'', M.Forti, P.
Alonso, P. Gargano, L. Kniznik, G. Rubiolo.
https://www.elsevier.com/events/conferences/the-nuclear-materials-conference

Noviembre 2016. 16º SAM-CONAMET. ``Shear stress in Fe/Fe 3 O4
interfaces''. Sociedad Argentina de Materiales.
\url{http://sam-conamet2016.congresos.unc.edu.ar/}

Octubre 2014. 14th SAM-CONAMET. ``Charge difference calculation in
Fe/Fe3O4 interfaces from DFT results''. Sociedad Argentina de
Materiales. http://www.unl.edu.ar/materiales2014/

Marzo 2014. Workshop en Procesamiento Físico Químico Avanzado. Workshop,
10 -- 15 March 2014, Universidad Nacional de Santander, Piedecuesta,
Colombia. Invited Speaker. i) ``Vasp Workshop'', b) ``Mechanical
Properties from DFT Calculations''.
http://www.sc3.uis.edu.co/pfqa-procesos-fisico-quimicos-avanzados/

Agosto 2013. 13th SAM-CONAMET. ``DFT Approximation to the Adhesion
Energy of the Fe(BCC)/Magnetite Interface''. Sociedad Argentina de
Materiales, Puerto Iguazú, Misiones, Argentina.

Noviembre 2012. XXXIX Reunión Anual de la Asociación Argentina de
Tecnología Nuclear. ``Atomistic Model of the Adhesion Problem in
Magnetite /iron system''. Asociación Argentina de Tecnología Nuclear.
Buenos Aires, Argentina.

Noviembre 2012. 4º Meeting of young researchers in Cience And
Technology. ``Spin-Orbit Coupling effect on bandstructure of Iron
monoxide in GGA+U approximation''. Sociedad Argentina de Materiales, Mar
del Plata, Buenos Aires, Argentina.

Octubre 2012. 12th CONAMET/SAM ``DFT study on adhesion energy in
Fe/Fe3O3 ubterface''. Universidad Técnica Federico Santa María,
CONAMET-SAM, Valparaíso, Chile. May 2012. 12th Anual Meeting of the
Nuclear Fuel Division. ``Predicting toughness in Fe/Magnetite interfaces
based in First Principle calculations''. Centro Atómico Constituyentes,
Comisión Nacional de Energía Atómica. Buenos Aires, Argentina.

Enero 2012. Pan American Advanced Institute, Computational Materials
Science for Energy Generation and Conversion. Santiago De Chile, Chile.
(Workshop, http://www.cnf.cornell.edu/cnf\_pasi2012.html)

Octubre 2011. SAM / CONAMET 2011, Rosario, Argentina. ``Ab Initio
Studies on carburization in Fe3Al based alloys'' (
http://www.ifir-conicet.gov.ar/SAM-ONAMET2011/documentos/topico6/215-161-1-SP.pdf).

Diciembre 2008. At The Frontiers of Condensed Matter IV. "Electric and
Magnetic properties of PMMA/Manganite composites".

Septiembre 2008 . Asociación de Física Argentina. "Electric and Magnetic
properties of PMMA/Manganite composites".

\section{Antecedentes como Formador}\label{antecedentes-como-formador}

Febrero de 2014. Co-Director de Pasantía. Diego Tozini, Instituto
Sabato. ``Automated edition of VASP outputs to calculate interface
interactions''. diegojosetozini@hotmail.com.

\section{Educación}\label{educaciuxf3n}

Septiembre 2017. Doctor en Ciencia y Tecnología, Mención Materiales.
Instituto de Tecnología Prof. Jorge Sabato, Universidad Nacional de
General San Martín. Director: Gerardo Rubiolo (rubiolo@cnea.gov.ar).

Título de la Tesis: Película pasivante en aceros de tubos de generadores
de vapor de centrales nucleares.

Resumen: El objetivo principal del trabajo es calcular la tenacidad a la
fractura de la interfaz α-Fe /(magnetita)Fe3O4 utilizando la Teoría del
Funcional de la Densidad (DFT). Se introducen solicitaciones de tracción
y de corte en el sistema para separar las partes de la interfaz. Los
cálculos de energía total se usan para investigar el comportamiento de
la interfaz y la influencia de las posiciones atómicas.

Agosto 2010. Ingeniero en Materiales. Instituto de Tecnología Prof.
Jorge Sabato, Universidad Nacional de General San Martín.

Trabajo Final de Ingeniería: ``Estudios Ab-Initio sobre carburación en
aleaciones de base Fe3Al''. Directores: Perla Balbuena
(balbuena@tamu.edu), Paula Alonso (pralonso@cnea.gov.ar) .

Resumen. Las aleaciones basadas en Fe-Al exhiben exelentes propiedades
pero sufren de pulverización en atmósferas carburizantes. La composición
de la superficie puede ser determinante para la solución del problema.
Se calcula las energías de adsorción de C en estructuras L21 Fe2AlX
(X=Ti,V,Nb) incluyendo la influencia de la cobertura. Los resultados
muestran un efecto beneficioso del Ti sugerida por la energía de
activación para la incorporación de C en las capas atómicas internas del
metal.

Idiomas: Inglés escrito y oral. Español nativo.

\section{Actividades Académicas}\label{actividades-acaduxe9micas}

Agosto 2012-2015. Representante de Egresados en Consejo Asesor de
Ingeniería en Materiales. Instituto Sabato, UNSAM-CNEA.

\section{Premios}\label{premios}

Mención Especial Estímulo a Jóvenes investigadores en Ciencia y
Tecnología de Materiales. Sociedad Argentina de Materiales, SAM-CONAMET
2011.

Finalista para la elección de la mejor Tesis Doctoral de la Universidad
de San Martín. Universidad de San Martín. 
