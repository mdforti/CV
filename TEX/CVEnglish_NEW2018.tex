\documentclass{my_cv}
\usepackage[utf8]{inputenc}
\usepackage{url}
\usepackage{titlesec}
\usepackage[]{hyperref}
\hypersetup{
    pdftitle={Resume},
    pdfauthor={Mariano Forti},
    pdfsubject={Mariano Forti Curriculum Vitae},
    pdfkeywords={Mariano Forti, Researcher, DFT, FEM, linux, programmer},
    bookmarksnumbered=false,     
    bookmarksopen=true,         
    bookmarksopenlevel=1,       
    colorlinks=false,            
    pdfstartview=Fit,           
    pdfpagemode=UseOutlines,    % this is the option you were lookin for
    pdfpagelayout=TwoPageRight
}

\urlstyle{same}
\begin{document}
\title{Mariano Daniel Forti}
\date{ \centering \url{mforti@cnea.gov.ar} \hspace{3cm} \url{marianodforti@gmail.com } }
\begin{titlepage}
\maketitle

\begin{abstract}
  \linespread{1.5}\selectfont
  \thispagestyle{empty}
    I am currently a researcher at Comisión Nacional de Energía Atómica, Argentina. My activities are carried on at División Aleaciones Especiales (Special Alloys Division) where I started working as a specialist in DFT calculations in 2017. My research is about mechanical stability of interfacial systems, mainly in the Iron/Magnetite interface and more recently the $Zr / ZrO_{2}$ system as well. I also participate actively in other activities carried on in the group, where we study point defects in $ZrO_2$.

Since my graduation as an Engineer I also work as a teaching assistant in an advanced course about the Finite Element Method for the Materials Engineering program at Instituto Sabato.

In September 2017 I completed the requirements to obtain the Doctorado en Ciencia y Tecnología (Doctorate in Science and Technology, Ph.D. ) at Universidad de San Martín. My degree studies are in Materials Engineering. Since the Final Work to get The Engineering degree I have been working in DFT calculations based on VASP, applied to surface and interfacial systems.

After obtaining my PhD in 2017, I started collaborating with the Special Alloy Foundry in Comisión de Energía Atómica in the preparations for the qualifications of the fabrication processes of alloys used in security components in Nuclear Power Plants. I also Assist the Foundry Staff in scientific basis for process optimization, quality assurance and technical decision taking. My responsibilities at DAE also include the maintenance of two private (though rather small) Linux computing clusters owned by DAE. Hence, during my professional career, I have acquired a wide experience in Linux System Administration for deskop and High Performance Computing Clusters. My programming skills include a variety of languages including Bash, Fortran, Python, php and javascript, and I also have wide experience in Matlab and Octave environments. I make a daily use of other tools as vim, OriginLab and gnuplot. In particular, I am an user and enthusiast of open source tools as the KDE desktop and the LibreOffice suite. 


\end{abstract}
\vfill
\end{titlepage}

\section{Profesional Experience}

\subsection{February 2017 - Current: Researcher} at Comisión Nacional de Energía Atómica,  Gerencia Materiales. División Aleaciones Especiales. \textbf{} Advisor: Dr. Paula Alonso (\url{pralonso@cnea.gov.ar}). Position: Researcher. Density Functional Theory calculations in interfacial systems: $Fe / Fe_3O_4$, $Zr/ZrO_2$. Point Defects in $ZrO_2$. Linux System Administration in Desktop and Cluster Computers for High Performance Calculations. Daily use of Bash, Fortran and Python. Scientific support for Special Alloy Foundry including quality assurance.

\subsection{August 2010 – Current: Teaching assistant} at Instituto Sabato (UNSaM\-CNEA), Ayudante de 1ra. “Computer Simulation of Processes and Materials”. Professor: Ruben Weht (\url{ruweht@cnea.gov.ar}). In this course students make their own implementation of several numerical methods for solving differential equations including Finite Differences and Finite Elements.

\subsection{August 2010 – January 2017. Full Time Scolarship} at Comisión Nacional de Energía Atómica. Gerencia Materiales. División Aleaciones Especiales. , “ Ab\-Initio studies about adhesion in iron/magnetite interfaces”. Advisor: Dr. Paula Alonso (\url{pralonso@cnea.gov.ar}). Research and development activities. Linux System Administration in Desktop and Cluster Computers for High Performance Calculations. Daily use of Bash, Fortran and Python. 

\subsection{August 2006- August 2010. Full Time Scolarship } at Instituto de Tecnología Jorge Sabato (ITJS, UNSAM-CNEA). Materials Engineering. 

\subsection{January 2010 – July 2010. Intern } at Texas A\&M University, Department of Chemical Engineering. Final Work to get the Materials Engineering Degree. “Ab-Initio Studies about Carburization of Fe3Al Based Alloys”. Advisor: Dr. Perla Balbuena (\url{balbuena@tamu.edu}). 

\subsection{February 2008. Internship }at Gerencia Física, CNEA. “Synthesis and Mechanical Properties of Manganite / Polymer composites”. Reference: Griselda Polla (\url{grispoll@cnea.gov.ar})

\subsection{February 2007. Internship} at Condensed Matter Group, CNEA. “Eléctric and Magnetic Properties of Manganites/Polymer Composites”. Reference: Joaquín Sacanell (\url{sacanell@cnea.gov.ar}). 

\section{Participation in research projects}

\subsection{February 2017 – February 2020: In the search for a new alloy for fuel elements of the CAREM-25 Power Plant}, PICT-2015-2267. Comisión Nacional de Energía Atómica, Agencia Nacional de Promoción Científica y Tecnológica, Ministerio de Ciencia, Técnica e innovación Productiva.

\subsection{January 2017 – December 2019: In the search for anewalloy for fuelel ements of the CAREM-25 Power Plant, }Project 80020160500046SM. InstitutoSabato, Universidad de San Martin, Comisión Nacional de Energía Atómica, Ministerio de Ciencia, Técnica e Innovación Productiva.

\subsection{August 2011- August 2015. Power Nuclear Reactor’s Materials integrity. Atomistic / Continuum models applied to inter-diffusion in disperse fuels and fracture in the oxide scales in steel pipes.} PICT-2011-1861 . Ministerio de Ciencia, Técnica e Innovación Productiva. 

\subsection{January 2011 – December 2012. Pasivating oxide scale over iron}, Project UNSAM C063. Universidad Nacional de San Martin. 

 \subsection{January 2011 – December 2011. Constitutional Defects and Aluminum Migration Energy in $UAl_4$}. Fundación Balseiro andComisión Nacional de Energía Atómica. 

\subsection{August 2010 – December 2015, Prespective studies, research and development of technologies for nuclear power plants of the fourth generation}. Comision Nacinoal de Energía Atómica.

 \subsection{January 2010 - December 2012. Computational Methods applied to the study of physical and chemical properties of fuel elements in research reactors, and modelization of defect cluster migrations in technologiacl materials}, CONICET.

\section{Publications}

\subsection{“Shear Behavior of $Fe/Fe_3O_4$ interfaces”. } Revista Materia V23-N2 (2018). Mariano Forti, Paula Alonso, Pablo Gargano, Gerardo Rubiolo. 

\subsection{“Properties of hexagonal $Zr$ and tetragonal $ZrO_2$ low index surfaces from DFT calculations”. } Revista Materia V23-N2 (2018). Paula Alonso, Pablo Gargano, Laura Kniznik, Gerardo Rubiolo. 

\subsection{“Concentration of constitutional and thermal defects in $UAl_4$ ” Journal of Nuclear Materials 478 (2016) 74-82. } Pablo Gargano, Laura Kniznik, Paula Alonso, Mariano Forti, Gerardo Rubiolo.

\subsection{“A DFT study of atomic structure and adhesion at the $Fe(BCC)/Fe_3 O_4$ interfaces”. } Surface Science 647 (2016) 55–65. Mariano Forti, Paula Alonso, Pablo Gargano,Perla Balbuena, Gerardo Rubiolo.

\subsection{“Charge difference calculation in $Fe/Fe_3O_4$ interfaces from DFT results”. } Procedia Materials Science 8 (2015) pp 1066 – 1072. Diego Tozini, Mariano Forti, Paula Alonso, Pablo Gargano, Gerardo Rubiolo.

\subsection{“Adhesion Energy of the $Fe(BCC)/Magnetite$ Interface within the DFT approach”. } Procedia Materials Science 9 (2015) pp 612 – 618. Diego Tozini, Mariano Forti, Pablo Gargano, Paula Alonso, Gerardo Rubiolo. 

\subsection{“Ab-initio studies on carburization of $Fe_3 Al$ based alloys”. } Procedia Materials Science 1 ( 2012 ) 191 – 198. Mariano Forti, Perla Balbuena, Paula Alonso.

 \subsection{“First principles study of $U-Al$ system ground state”. } Procedia Materials Science 1 ( 2012 ) 514 –519. Laura Kniznik, Paula R. Alonso, Pablo H. Gargano, Mariano D. Forti, Gerardo H. Rubiolo.

\subsection{“Transition metals monoxides. } An LDA+U study”. Procedia Materials Science 1 ( 2012 ) 230 – 234. Mariano Forti, Paula R. Alonso, Pablo H. Gargano, Gerardo H. Rubiolo.

\subsection{“Electric and magnetic properties of PMMA/manganite composites”. } Physica B 404 (2009) 2760– 2762. Artale, C., Fermepin, S., Latino, M., Quintero, M., Granja, L., Sacanell, J., Polla G., Levy P. 

\section{Participation in Congress}

\subsection{ 5th Nuclear Materials Conference, Elsevier – IAEA, 14-18 October 2018} Seattle, USA. Poster Presentations:
\emph{ 1) “First-principles thermodynamic study of point-defect structure and electrical conductivity in tetragonal non-stoichiometric zirconia including lattice vibrations”} , Gargano, Kniznik, Alonso, Forti, Rubuiolo. \emph{ 2) “DFT Study of the Early Stages of Oxidation of the Zr(1010) Surface”} , F. Soto, M. Forti, P. Alonso, P. Gargano, L. Kniznik, G. Rubiolo, P. Balbuena. \emph{ 3) “A DFT study of the resistance to traction and shear loads of the $Fe(BCC) / Fe_3O_4$ interface”}, M.Forti, P. Alonso, P. Gargano, L. Kniznik, G. Rubiolo. 

\subsection{16º SAM-CONAMET, November 2016} Córdoba, Argentina. \emph{“Shear stress in Fe/Fe 3 O4 interfaces”. }Sociedad Argentina de Materiales. \url{http://sam-conamet2016.congresos.unc.edu.ar/}

\subsection{14th SAM-CONAMET, October 2014,} Santa Fe, Argentina. “Charge difference calculation in $Fe/Fe_3O_4$ interfaces from DFT results”. Sociedad Argentina de Materiales. \url{http://www.unl.edu.ar/materiales2014/}

\subsection{ Workshop en Procesamiento Físico Químico Avanzado. 11 – 15 March 2014, Universidad Nacional de Santander}, Piedecuesta, Colombia. Invited speaker. \emph{ a) “Vasp Workshop”},\emph{ b) “Mechanical Properties from DFT Calculations”}. 
 http://www.sc3.uis.edu.co/pfqa-procesos-fisico-quimicos-avanzados/.

\subsection{13th SAM-CONAMET, August 2013.} Sociedad Argentina de Materiales, Puerto Iguazú, Misiones, Argentina “DFT Approximation to the Adhesion Energy of the Fe(BCC)/Magnetite Interface”..

\subsection{XXXIX Reunión Anual de la Asociación Argentina de Tecnología Nuclear, November 2012.}, Buenos Aires, Argentina.
\emph{ “Atomistic Model of the Adhesion Problem in Magnetite /iron system”. }

\subsection{4º Meeting of young researchers in Cience And Technology, October 2012.}Sociedad Argentina de Materiales, Mar del Plata, Buenos Aires, Argentina. \emph{  “Spin-Orbit Coupling effect on bandstructure of Iron monoxide in GGA+U approximation”. }

\subsection{12th CONAMET/SAM, October 2012.}  Universidad Técnica Federico Santa María, CONAMET-SAM, Valparaíso, Chile.\emph{“DFT study on adhesion energy in Fe/Fe3O3 ubterface”.}

\subsection{May 2012. 12th Anual Meeting of the Nuclear Fuel Division.} Centro Atómico Constituyentes, Comisión Nacional de Energía Atómica. Buenos Aires, Argentina.\emph{“Predicting toughness in Fe/Magnetite interfaces based in First Principle calculations”.}

\subsection{PASI, January 2012.} Pan American Advanced Institute, Computational Materials Science for Energy Generation and Conversion. Santiago De Chile, Chile. (Workshop, \url{http://www.cnf.cornell.edu/cnf\_pasi2012.html} ) 

\subsection{SAM / CONAMET 2011, October 2011.}  Rosario, Argentina. \emph{“Ab Initio Studies on carburization in Fe3Al based alloys”} (\url{http://www.ifir-conicet.gov.ar/SAM-ONAMET2011/documentos/topico6/215-161-1-SP.pdf} ). 

\subsection{At The Frontiers of Condensed Matter IV, December 2008.} Centro Atómico Constituyentes, Buenos Aires, Argentina. \emph{"Electric and Magnetic properties of PMMA/Manganite composites". }

\subsection{RAFA, September 2008.} Reunión Anual de la Asociación de Física Argentina, Salta, Argentina. \emph{"Electric and Magnetic properties of PMMA/Manganite composites". }

\section{Advisories}

\subsection{ Co-Advisor of intern, February 2014.} Diego Tozini, Instituto Sabato. “Automated edition
of VASP outputs to calculate interface interactions”. \url{diegojosetozini@hotmail.com
}.\section{Education}

\subsection{Doctor in Materials Science and Technology (PhD), September 2017.} Instituto de
Tecnología Prof. Jorge Sabato, Universidad Nacional de General San Martín. \subsubsection{Advisor:}
Gerardo Rubiolo (\url{rubiolo@cnea.gov.ar}).
\subsubsection{PhD Thesis title:} Pasivating film in steel pipes used in nuclear power plants steam
generators.
\subsubsection{Abstract:} The main objective of this work was to calculate the fracture toughness of the $\alpha-Fe$ /(magnetite)Fe3O4 interface using Density Functional Theory (DFT). Traction and
shear mode strains are introduced to the system to separate the parts, and total energy
calculations are used to investigate the behavior of the interface and the influence of
atomic arrangement.

\subsection{Materlials Engineer, August 2010.} Instituto de Tecnología Prof. Jorge Sabato,
Universidad Nacional de General San Martín.
\subsubsection{Title of the Engineering dissertation:} “Ab-Initio Studies about Carburization of Fe3Al Based Alloys”.
\subsubsection{Advisors:} Perla Balbuena (\url{balbuena@tamu.edu}), Paula Alonso (\url{pralonso@cnea.gov.ar})
\subsubsection{Abstract:} Fe-Al based alloys exhibit excellent properties but suffer metal dusting in
carburizing atmospheres. Surface composition can be a determinant factor in the solution
of this problem. We calculate in this work the C adsorption energies in the L21 Fe2AlX
(X=Ti,V,Nb) structures and we study the influence of surface covering. Our results show
the beneficial effect of Ti, suggesting that there could exist an activation energy to promote
the incorporation of C in the subsurface layers when the surface is covered enough.
\subsection{Languages.} oral and written English, Native Spanish.

\section{Other Academic Activities.}

\subsection{August 2012- August 2015.} Member of the Counseling Board as a representative of
former students of Materials Engineering, Instituto Sabato, UNSAM-CNEA.

\section{Awards}

\subsection{2011} Special Mention to Young Researchers in Materials Cience and Technology. Sociedad
Argentina de Materiales, SAM-CONAMET.

\subsection{2018} Special Mention as a Finalist for Best Doctoral Thesis Awards, Universidad de San
Martín. 

\end{document}
