%\begin{titlepage}
%\maketitle
\begin{abstract}
  \linespread{1.5}\selectfont
  \thispagestyle{empty}

  I have recently started a postdoctoral position at the Interdisciplinary Center of Advanced Materials 
  Simulations (ICMAS),  under the supervision of Dr. Prof. Ralph Drautz and Dr. habil. Thomas 
  Hammerschmidt. There, I will carry on investigations about Niquel based Superalloys using 
  Datascience and Machine learning techinques. 

  Since 2017 I have been a researcher at  Comisión Nacional de Energía
  Atómica, Argentina. My activities are carried on at División Aleaciones
  Especiales (Special Alloys Division) where I started working as a
  specialist in DFT calculations in 2017. My research is about mechanical
  stability of interfacial systems, mainly in the Iron/Magnetite interface
  and more recently the Zr / ZrO\textsubscript{2} system as well. I also
  participate actively in other activities carried on in the group, where we
  study point defects in ZrO\textsubscript{2}.

  My responsibilities at DAE also include the maintenance of two private (though
  rather small) Linux computing clusters owned by DAE. Hence, during my
  professional career, I have acquired a wide experience in Linux System
  Administration for deskop and High Performance Computing Clusters. My
  programming skills include a variety of languages including Bash, Fortran,
  Python, php and javascript, and I also have wide experience in Matlab and
  Octave environments. I make a daily use of other tools as vim, OriginLab and
  gnuplot. In particular, I am an user and enthusiast of open source tools as
  the KDE desktop and the LibreOffice suite. 

  In September 2017 I completed the requirements to obtain the Doctorado en
  Ciencia y Tecnología (Doctorate in Science and Technology, Ph.D. ) at
  Universidad de San Martín. My degree studies are in Materials Engineering.
  Since the Final Work to get The Engineering degree I have been working in DFT
  calculations based on VASP, applied to surface and interfacial systems.

  I also work as a teaching assistant in an advanced course about the Finite
  Element Method for the Materials Engineering program at Instituto Sabato.

\end{abstract}
\vfill
%\end{titlepage}
