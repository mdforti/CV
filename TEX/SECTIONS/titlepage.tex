%\begin{titlepage}
%\maketitle
\begin{abstract}
  \linespread{1.5}\selectfont
  \thispagestyle{empty}
    I am currently a researcher at Comisión Nacional de Energía Atómica, Argentina. My activities are carried on at División Aleaciones Especiales (Special Alloys Division) where I started working as a specialist in DFT calculations in 2017. My research is about mechanical stability of interfacial systems, mainly in the Iron/Magnetite interface and more recently the $Zr / ZrO_{2}$ system as well. I also participate actively in other activities carried on in the group, where we study point defects in $ZrO_2$.
 
  My responsibilities at DAE also include the maintenance of two private (though rather small) Linux computing clusters owned by DAE. Hence, during my professional career, I have acquired a wide experience in Linux System Administration for deskop and High Performance Computing Clusters. My programming skills include a variety of languages including Bash, Fortran, Python, php and javascript, and I also have wide experience in Matlab and Octave environments. I make a daily use of other tools as vim, OriginLab and gnuplot. In particular, I am an user and enthusiast of open source tools as the KDE desktop and the LibreOffice suite. 

In September 2017 I completed the requirements to obtain the Doctorado en Ciencia y Tecnología (Doctorate in Science and Technology, Ph.D. ) at Universidad de San Martín. My degree studies are in Materials Engineering. Since the Final Work to get The Engineering degree I have been working in DFT calculations based on VASP, applied to surface and interfacial systems.

After obtaining my PhD in 2017, I started collaborating with the Special Alloy Foundry in Comisión de Energía Atómica in the preparations for the qualifications of the fabrication processes of alloys used in security components in Nuclear Power Plants. I also Assist the Foundry Staff in scientific basis for process optimization, quality assurance and technical decision taking. 

  Since my graduation as an Engineer I also work as a teaching assistant in an advanced course about the Finite Element Method for the Materials Engineering program at Instituto Sabato.

\end{abstract}
\vfill
%\end{titlepage}
