\section{Participation in Congress}

\subsection{ 5th Nuclear Materials Conference, Elsevier – IAEA, 14-18 October 2018} Seattle, USA. Poster Presentations:
\emph{ 1) “First-principles thermodynamic study of point-defect structure and electrical conductivity in tetragonal non-stoichiometric zirconia including lattice vibrations”} , Gargano, Kniznik, Alonso, Forti, Rubuiolo. \emph{ 2) “DFT Study of the Early Stages of Oxidation of the Zr(1010) Surface”} , F. Soto, M. Forti, P. Alonso, P. Gargano, L. Kniznik, G. Rubiolo, P. Balbuena. \emph{ 3) “A DFT study of the resistance to traction and shear loads of the Fe(BCC) / Fe\textsubscript{3}O\textsubscript{4} interface”}, M.Forti, P. Alonso, P. Gargano, L. Kniznik, G. Rubiolo. 
\url{https://elsevier.conference-services.net/programme.asp?conferenceID=4229&action=prog_list&session=45340}

\subsection{16º SAM-CONAMET, November 2016} Córdoba, Argentina. \emph{“Shear stress in Fe/Fe 3 O4 interfaces”. }Sociedad Argentina de Materiales. \url{http://sam-conamet2016.congresos.unc.edu.ar/}

\subsection{14th SAM-CONAMET, October 2014,} Santa Fe, Argentina. “Charge difference calculation in Fe/Fe\textsubscript{3}O\textsubscript{4} interfaces from DFT results”. Sociedad Argentina de Materiales. \url{http://www.unl.edu.ar/materiales2014/}

\subsection{ Workshop en Procesamiento Físico Químico Avanzado. 11 – 15 March 2014, Universidad Nacional de Santander}, Piedecuesta, Colombia. Invited speaker. \emph{ a) “Vasp Workshop”},\emph{ b) “Mechanical Properties from DFT Calculations”}. 
\burl{http://www.sc3.uis.edu.co/pfqa-procesos-fisico-quimicos-avanzados/}.

\subsection{13th SAM-CONAMET, August 2013.} Sociedad Argentina de Materiales, Puerto Iguazú, Misiones, Argentina “DFT Approximation to the Adhesion Energy of the Fe(BCC)/Magnetite Interface”..

\subsection{XXXIX Reunión Anual de la Asociación Argentina de Tecnología Nuclear, November 2012.}, Buenos Aires, Argentina.
\emph{ “Atomistic Model of the Adhesion Problem in Magnetite /iron system”. }

\subsection{4º Meeting of young researchers in Cience And Technology, October 2012.}Sociedad Argentina de Materiales, Mar del Plata, Buenos Aires, Argentina. \emph{  “Spin-Orbit Coupling effect on bandstructure of Iron monoxide in GGA+U approximation”. }

\subsection{12th CONAMET/SAM, October 2012.}  Universidad Técnica Federico Santa María, CONAMET-SAM, Valparaíso, Chile.\emph{“DFT study on adhesion energy in Fe/Fe\textsubscript{3}O\textsubscript{4} interface”.}

\subsection{May 2012. 12th Anual Meeting of the Nuclear Fuel Division.} Centro Atómico Constituyentes, Comisión Nacional de Energía Atómica. Buenos Aires, Argentina.\emph{“Predicting toughness in Fe/Magnetite interfaces based in First Principle calculations”.}

\subsection{PASI, January 2012.} Pan American Advanced Institute, Computational Materials Science for Energy Generation and Conversion. Santiago De Chile, Chile. (Workshop, \url{http://www.cnf.cornell.edu/cnf\_pasi2012.html} ) 

\subsection{SAM / CONAMET 2011, October 2011.}  Rosario, Argentina. \emph{“Ab Initio Studies on carburization in Fe\textsubscript{3}Al based alloys”} (\url{http://www.ifir-conicet.gov.ar/SAM-ONAMET2011/documentos/topico6/215-161-1-SP.pdf} ). 

\subsection{At The Frontiers of Condensed Matter IV, December 2008.} Centro Atómico Constituyentes, Buenos Aires, Argentina. \emph{"Electric and Magnetic properties of PMMA/Manganite composites". }

\subsection{RAFA, September 2008.} Reunión Anual de la Asociación de Física Argentina, Salta, Argentina. \emph{"Electric and Magnetic properties of PMMA/Manganite composites". }

