\section{Education}

\subsection{Doctor in Materials Science and Technology (PhD), September 2017.} Instituto de
Tecnología Prof. Jorge Sabato, Universidad Nacional de General San Martín. \subsubsection{Adviser:}
Gerardo Rubiolo (\url{rubiolo@cnea.gov.ar}).
\subsubsection{PhD Thesis title:} Pasivating film in steel pipes used in nuclear power plants steam
generators.
\subsubsection{Abstract:} The main objective of this work was to calculate the fracture toughness of the $\alpha$-Fe /(magnetite)Fe3O4 interface using Density Functional Theory (DFT). Traction and
shear mode strains are introduced to the system to separate the parts, and total energy
calculations are used to investigate the behavior of the interface and the influence of
atomic arrangement.

\subsection{Materlials Engineer, August 2010.} Instituto de Tecnología Prof. Jorge Sabato,
Universidad Nacional de General San Martín.
\subsubsection{Title of the Engineering dissertation:} “Ab-Initio Studies about Carburization of Fe3Al Based Alloys”.
\subsubsection{Advisers:} Perla Balbuena (\url{balbuena@tamu.edu}), Paula Alonso (\url{pralonso@cnea.gov.ar})
\subsubsection{Abstract:} Fe-Al based alloys exhibit excellent properties but suffer metal dusting in
carburizing atmospheres. Surface composition can be a determinant factor in the solution
of this problem. We calculate in this work the C adsorption energies in the L21 Fe2AlX
(X=Ti,V,Nb) structures and we study the influence of surface covering. Our results show
the beneficial effect of Ti, suggesting that there could exist an activation energy to promote
the incorporation of C in the subsurface layers when the surface is covered enough.
\subsection{Languages.} oral and written English, Native Spanish.

